% Use only LaTeX2e, calling the article.cls class and 12-point type.

\documentclass[12pt]{article}

% Users of the {thebibliography} environment or BibTeX should use the
% scicite.sty package, downloadable from *Science* at
% www.sciencemag.org/about/authors/prep/TeX_help/ .
% This package should properly format in-text
% reference calls and reference-list numbers.
\usepackage{cite}
\usepackage{amsmath}
\usepackage{lineno}
\linenumbers

% Use times if you have the font installed; otherwise, comment out the
% following line.

\usepackage{times}

% The preamble here sets up a lot of new/revised commands and
% environments.  It's annoying, but please do *not* try to strip these
% out into a separate .sty file (which could lead to the loss of some
% information when we convert the file to other formats).  Instead, keep
% them in the preamble of your main LaTeX source file.


% The following parameters seem to provide a reasonable page setup.

\topmargin 0.0cm
\oddsidemargin 0.2cm
\textwidth 16cm 
\textheight 21cm
\footskip 1.0cm


%The next command sets up an environment for the abstract to your paper.

\newenvironment{sciabstract}{%
\begin{quote} \bf}
{\end{quote}}


% If your reference list includes text notes as well as references,
% include the following line; otherwise, comment it out.

\renewcommand\refname{References and Notes}

% The following lines set up an environment for the last note in the
% reference list, which commonly includes acknowledgments of funding,
% help, etc.  It's intended for users of BibTeX or the {thebibliography}
% environment.  Users who are hand-coding their references at the end
% using a list environment such as {enumerate} can simply add another
% item at the end, and it will be numbered automatically.

\newcounter{lastnote}
\newenvironment{scilastnote}{%
\setcounter{lastnote}{\value{enumiv}}%
\addtocounter{lastnote}{+1}%
\begin{list}%
{\arabic{lastnote}.}
{\setlength{\leftmargin}{.22in}}
{\setlength{\labelsep}{.5em}}}
{\end{list}}


% Include your paper's title here

\title{Study of low multiplicity subtraction of jet contribution on Fourrier harmonics} 


% Place the author information here.  Please hand-code the contact
% information and notecalls; do *not* use \footnote commands.  Let the
% author contact information appear immediately below the author names
% as shown.  We would also prefer that you don't change the type-size
% settings shown here.

\author
{Maxime Guilbaud,$^{1,a}$ Wei Li$^{1,b}$, Zhenyu Chen$^{1,c}$\\
\\
\normalsize{$^{1}$Bonner Laboratory, Department of Physics And Astronomy, RICE University,}\\
\normalsize{6100 Main MS-550, Houston, TX 77005-1827, USA}\\
\\
\normalsize{$^{a}$m.guilbaud@cern.ch, }\\
\normalsize{$^{b}$davidlw@rice.edu}\\
\normalsize{$^{c}$zhenyu.chen@cern.ch}\\
}

% Include the date command, but leave its argument blank.

\date{}

%%%%%%%%%%%%%%%%% END OF PREAMBLE %%%%%%%%%%%%%%%%



\begin{document} 

% Double-space the manuscript.

\baselineskip24pt

% Make the title.

\maketitle 

% Place your abstract within the special {sciabstract} environment.

\begin{abstract}

Some abstract there. The first ridge discovery in p-p is here: Ref.~\cite{Khachatryan:2010gv}

\end{abstract}

\input{srcs/introduction}
\section{ATLAS template fit method}

A new template fit function is used by ATLAS~\cite{Aad:2015gqa} to extract elliptical flow harmonic $v_{2}$ from two-particle correlation in pp collisions, as 

\begin{equation}
\label{eq:ATLAStemplate}
Y^{templ}(\Delta\phi) = F Y^{periph}(\Delta\phi) + Y^{ridge}(\Delta\phi),
\end{equation}

where

\begin{equation}
\label{eq:ATLAStemplateridge}
Y^{ridge}(\Delta\phi) = G(1+2V_{2\Delta}^{fit}cos(2\Delta\phi)).
\end{equation}

Here $Y(\Delta\phi)$ is the 1D per-trigger-particle yield extracted from 2D correlation functions and 
$V_{2\Delta}^{fit}$ is the collective flow signal extracted.
The per-trigger-particle yield $Y(\Delta\phi)$ can be Fourier decomposed,

\begin{equation}
\label{eq:ATLASyieldVn}
Y(\Delta\phi)  = N \left\{1+\sum\limits_{n} 2V_{n\Delta} \cos (n\Delta\phi)\right\}.
\end{equation}

Expand Eq.~\ref{eq:ATLAStemplate} using Fourier decomposition, relations between various variables can be extracted,

\begin{equation}
\label{eq:ATLASfit1}
G = N - FN^{periph},
\end{equation}

\begin{linenomath}
\begin{equation}
\label{eq:ATLASfit2}
GV_{n\Delta}^{fit} = NV_{n\Delta} - FN^{periph}V_{n\Delta}^{periph}.
\end{equation}
\end{linenomath}

Previous ATLAS pPb analyses use the peripheral-subtraction method with the peripheral reference after applying the ZYAM procedure,
i.e. $Y^{periph}(\Delta\phi) - Y^{periph}(0)$.
The ZYAM procedure is similar to subtract out the 0th order term in the Fourier decomposition, $FN^{periph}$ in Eq.~\ref{eq:ATLASfit1},
which explicitly forces $G$ to be equal to $N$. As $N$ represents the number of particle pairs involved in the correlation,
$G = N$ implies that the extracted $V_{n\Delta}$ and $V_{n\Delta}^{fit}$ are from the same group of particles.
A similar procedure has also been used by the CMS Collaboration~\cite{Chatrchyan:2013nka}.

In ATLAS's new method, $G$ is required to be smaller than $N$, 
which would indicate that $V_{n\Delta}^{fit}$ quantifies the correlation of a smaller set of particle pairs,
although it is not clear how to distinguish sets of particles involved in collective correlation with those involved in few-body correlation.
Therefore, comparison between anisotropy harmonics results is not meaningful as there is fundamental difference in the definition.
%In particular, when applied to events with similar multiplicity as the peripheral reference, 
%$V_{n\Delta}^{fit}$ quantifies correlation of very few particles, since $G$ is expected to be very small.


%\section{CMS method}
\input{srcs/summary}

\bibliography{theNote}
\bibliographystyle{plain}

\clearpage

\end{document}




















